\documentclass{report}
\usepackage{hyperref}
% WARNING: THIS SHOULD BE MODIFIED DEPENDING ON THE LETTER/A4 SIZE
\oddsidemargin 0cm
\evensidemargin 0cm
\marginparsep 0cm
\marginparwidth 0cm
\parindent 0cm
\setlength{\textwidth}{\paperwidth}
\addtolength{\textwidth}{-2in}


% Conditional define to determine if pdf output is used
\newif\ifpdf
\ifx\pdfoutput\undefined
\pdffalse
\else
\pdfoutput=1
\pdftrue
\fi

\ifpdf
  \usepackage[pdftex]{graphicx}
\else
  \usepackage[dvips]{graphicx}
\fi

% Write Document information for pdflatex/pdftex
\ifpdf
\pdfinfo{
 /Author     (Pasdoc)
 /Title      ()
}
\fi


% definitons for warning and note tag
\usepackage[most]{tcolorbox}
\newtcolorbox{tcbwarning}{
 breakable,
 enhanced jigsaw,
 top=0pt,
 bottom=0pt,
 titlerule=0pt,
 bottomtitle=0pt,
 rightrule=0pt,
 toprule=0pt,
 bottomrule=0pt,
 colback=white,
 arc=0pt,
 outer arc=0pt,
 title style={white},
 fonttitle=\color{black}\bfseries,
 left=8pt,
 colframe=red,
 title={Warning:},
}
\newtcolorbox{tcbnote}{
 breakable,
 enhanced jigsaw,
 top=0pt,
 bottom=0pt,
 titlerule=0pt,
 bottomtitle=0pt,
 rightrule=0pt,
 toprule=0pt,
 bottomrule=0pt,
 colback=white,
 arc=0pt,
 outer arc=0pt,
 title style={white},
 fonttitle=\color{black}\bfseries,
 left=8pt,
 colframe=yellow,
 title={Note:},
}

\begin{document}
\label{toc}\tableofcontents
\newpage
% special variable used for calculating some widths.
\newlength{\tmplength}
\chapter{Unit ok{\_}enum{\_}explicit{\_}assign}
\label{ok_enum_explicit_assign}
\index{ok{\_}enum{\_}explicit{\_}assign}
\section{Types}
\ifpdf
\subsection*{\large{\textbf{TEnum1}}\normalsize\hspace{1ex}\hrulefill}
\else
\subsection*{TEnum1}
\fi
\label{ok_enum_explicit_assign-TEnum1}
\index{TEnum1}
\begin{list}{}{
\settowidth{\tmplength}{\textbf{Description}}
\setlength{\itemindent}{0cm}
\setlength{\listparindent}{0cm}
\setlength{\leftmargin}{\evensidemargin}
\addtolength{\leftmargin}{\tmplength}
\settowidth{\labelsep}{X}
\addtolength{\leftmargin}{\labelsep}
\setlength{\labelwidth}{\tmplength}
}
\item[\textbf{Declaration}\hfill]
\ifpdf
\begin{flushleft}
\fi
\begin{ttfamily}
TEnum1 = (...);\end{ttfamily}

\ifpdf
\end{flushleft}
\fi

\par
\item[\textbf{Description}]
 \item[\textbf{Values}]
\begin{description}
\item[\texttt{e1One}] \label{ok_enum_explicit_assign-e1One}
\index{}
 
\item[\texttt{e1Two = 12}] \label{ok_enum_explicit_assign-e1Two}
\index{}
 
\item[\texttt{e1Three}] \label{ok_enum_explicit_assign-e1Three}
\index{}
 
\item[\texttt{e1Four := 15}] \label{ok_enum_explicit_assign-e1Four}
\index{}
 
\end{description}


\end{list}
\ifpdf
\subsection*{\large{\textbf{TEnum2}}\normalsize\hspace{1ex}\hrulefill}
\else
\subsection*{TEnum2}
\fi
\label{ok_enum_explicit_assign-TEnum2}
\index{TEnum2}
\begin{list}{}{
\settowidth{\tmplength}{\textbf{Description}}
\setlength{\itemindent}{0cm}
\setlength{\listparindent}{0cm}
\setlength{\leftmargin}{\evensidemargin}
\addtolength{\leftmargin}{\tmplength}
\settowidth{\labelsep}{X}
\addtolength{\leftmargin}{\labelsep}
\setlength{\labelwidth}{\tmplength}
}
\item[\textbf{Declaration}\hfill]
\ifpdf
\begin{flushleft}
\fi
\begin{ttfamily}
TEnum2 = (...);\end{ttfamily}

\ifpdf
\end{flushleft}
\fi

\par
\item[\textbf{Description}]
 \item[\textbf{Values}]
\begin{description}
\item[\texttt{e2One := 3}] \label{ok_enum_explicit_assign-e2One}
\index{}
 
\item[\texttt{e1Two := 4}] \label{ok_enum_explicit_assign-e1Two}
\index{}
 
\end{description}


\end{list}
\end{document}
