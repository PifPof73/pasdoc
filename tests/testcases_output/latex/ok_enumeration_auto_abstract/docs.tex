\documentclass{report}
\usepackage{hyperref}
% WARNING: THIS SHOULD BE MODIFIED DEPENDING ON THE LETTER/A4 SIZE
\oddsidemargin 0cm
\evensidemargin 0cm
\marginparsep 0cm
\marginparwidth 0cm
\parindent 0cm
\setlength{\textwidth}{\paperwidth}
\addtolength{\textwidth}{-2in}


% Conditional define to determine if pdf output is used
\newif\ifpdf
\ifx\pdfoutput\undefined
\pdffalse
\else
\pdfoutput=1
\pdftrue
\fi

\ifpdf
  \usepackage[pdftex]{graphicx}
\else
  \usepackage[dvips]{graphicx}
\fi

% Write Document information for pdflatex/pdftex
\ifpdf
\pdfinfo{
 /Author     (Pasdoc)
 /Title      ()
}
\fi


% definitons for warning and note tag
\usepackage[most]{tcolorbox}
\newtcolorbox{tcbwarning}{
 breakable,
 enhanced jigsaw,
 top=0pt,
 bottom=0pt,
 titlerule=0pt,
 bottomtitle=0pt,
 rightrule=0pt,
 toprule=0pt,
 bottomrule=0pt,
 colback=white,
 arc=0pt,
 outer arc=0pt,
 title style={white},
 fonttitle=\color{black}\bfseries,
 left=8pt,
 colframe=red,
 title={Warning:},
}
\newtcolorbox{tcbnote}{
 breakable,
 enhanced jigsaw,
 top=0pt,
 bottom=0pt,
 titlerule=0pt,
 bottomtitle=0pt,
 rightrule=0pt,
 toprule=0pt,
 bottomrule=0pt,
 colback=white,
 arc=0pt,
 outer arc=0pt,
 title style={white},
 fonttitle=\color{black}\bfseries,
 left=8pt,
 colframe=yellow,
 title={Note:},
}

\begin{document}
\label{toc}\tableofcontents
\newpage
% special variable used for calculating some widths.
\newlength{\tmplength}
\chapter{Unit ok{\_}enumeration{\_}auto{\_}abstract}
\label{ok_enumeration_auto_abstract}
\index{ok{\_}enumeration{\_}auto{\_}abstract}
\section{Types}
\ifpdf
\subsection*{\large{\textbf{TEnum}}\normalsize\hspace{1ex}\hrulefill}
\else
\subsection*{TEnum}
\fi
\label{ok_enumeration_auto_abstract-TEnum}
\index{TEnum}
\begin{list}{}{
\settowidth{\tmplength}{\textbf{Description}}
\setlength{\itemindent}{0cm}
\setlength{\listparindent}{0cm}
\setlength{\leftmargin}{\evensidemargin}
\addtolength{\leftmargin}{\tmplength}
\settowidth{\labelsep}{X}
\addtolength{\leftmargin}{\labelsep}
\setlength{\labelwidth}{\tmplength}
}
\item[\textbf{Declaration}\hfill]
\ifpdf
\begin{flushleft}
\fi
\begin{ttfamily}
TEnum = (...);\end{ttfamily}

\ifpdf
\end{flushleft}
\fi

\par
\item[\textbf{Description}]
Enum type.\item[\textbf{Values}]
\begin{description}
\item[\texttt{en1}] \label{ok_enumeration_auto_abstract-en1}
\index{}
Enum value 1
\item[\texttt{en2}] \label{ok_enumeration_auto_abstract-en2}
\index{}
Enum value 2.
\item[\texttt{en3}] \label{ok_enumeration_auto_abstract-en3}
\index{}
Enum value 3. With longer description
\item[\texttt{en4}] \label{ok_enumeration_auto_abstract-en4}
\index{}
Enum value 4. With longer description.
\item[\texttt{en5}] \label{ok_enumeration_auto_abstract-en5}
\index{}
Enum value 5.

With longer description
\item[\texttt{en6}] \label{ok_enumeration_auto_abstract-en6}
\index{}
Enum value 6.

With longer description.
\end{description}


\end{list}
\end{document}
