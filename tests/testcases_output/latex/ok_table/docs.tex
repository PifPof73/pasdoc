\documentclass{report}
\usepackage{hyperref}
% WARNING: THIS SHOULD BE MODIFIED DEPENDING ON THE LETTER/A4 SIZE
\oddsidemargin 0cm
\evensidemargin 0cm
\marginparsep 0cm
\marginparwidth 0cm
\parindent 0cm
\setlength{\textwidth}{\paperwidth}
\addtolength{\textwidth}{-2in}


% Conditional define to determine if pdf output is used
\newif\ifpdf
\ifx\pdfoutput\undefined
\pdffalse
\else
\pdfoutput=1
\pdftrue
\fi

\ifpdf
  \usepackage[pdftex]{graphicx}
\else
  \usepackage[dvips]{graphicx}
\fi

% Write Document information for pdflatex/pdftex
\ifpdf
\pdfinfo{
 /Author     (Pasdoc)
 /Title      ()
}
\fi


% definitons for warning and note tag
\usepackage[most]{tcolorbox}
\newtcolorbox{tcbwarning}{
 breakable,
 enhanced jigsaw,
 top=0pt,
 bottom=0pt,
 titlerule=0pt,
 bottomtitle=0pt,
 rightrule=0pt,
 toprule=0pt,
 bottomrule=0pt,
 colback=white,
 arc=0pt,
 outer arc=0pt,
 title style={white},
 fonttitle=\color{black}\bfseries,
 left=8pt,
 colframe=red,
 title={Warning:},
}
\newtcolorbox{tcbnote}{
 breakable,
 enhanced jigsaw,
 top=0pt,
 bottom=0pt,
 titlerule=0pt,
 bottomtitle=0pt,
 rightrule=0pt,
 toprule=0pt,
 bottomrule=0pt,
 colback=white,
 arc=0pt,
 outer arc=0pt,
 title style={white},
 fonttitle=\color{black}\bfseries,
 left=8pt,
 colframe=yellow,
 title={Note:},
}

\begin{document}
\label{toc}\tableofcontents
\newpage
% special variable used for calculating some widths.
\newlength{\tmplength}
\chapter{Unit ok{\_}table}
\label{ok_table}
\index{ok{\_}table}
\section{Description}
Test of @table{-}related features.\hfill\vspace*{1ex}



Example from @table doc page in wiki:



\begin{tabular}{|l|l|l|}
\hline
\textbf{Value1} & \textbf{Value2} & \textbf{Result} \\ \hline
\begin{ttfamily}False\end{ttfamily} & \begin{ttfamily}False\end{ttfamily} & \begin{ttfamily}False\end{ttfamily} \\ \hline
\begin{ttfamily}False\end{ttfamily} & \begin{ttfamily}True\end{ttfamily} & \begin{ttfamily}True\end{ttfamily} \\ \hline
\begin{ttfamily}True\end{ttfamily} & \begin{ttfamily}False\end{ttfamily} & \begin{ttfamily}True\end{ttfamily} \\ \hline
\begin{ttfamily}True\end{ttfamily} & \begin{ttfamily}True\end{ttfamily} & \begin{ttfamily}False\end{ttfamily} \\ \hline
\end{tabular}



Small tables tests:



\begin{tabular}{|l|}
\hline
One{-}cell table is OK \\ \hline
\end{tabular}

 

\begin{tabular}{|l|}
\hline
\textbf{One{-}cell head table is OK} \\ \hline
\end{tabular}

 

\begin{tabular}{|l|l|}
\hline
\textbf{A} & \textbf{ \textbf{Foo} } \\ \hline
\end{tabular}



Test that everything within @cell tag is OK. Actually, test below is a stripped down (to be accepted by LaTeX) version of a more advanced test in ok{\_}table{\_}nonlatex.pas file.



\begin{tabular}{|l|l|}
\hline
\textbf{Dashes: ---, --, {-}, {-}{-}} & \textbf{URLs: \href{http://pasdoc.sourceforge.net/}{http://pasdoc.sourceforge.net/}} \\ \hline
C & D \\ \hline
\end{tabular}



Now nested table and other nicies are within a normal row, instead of heading row.



\begin{tabular}{|l|l|}
\hline
\textbf{C} & \textbf{D} \\ \hline
Within a cell many things are are accepted & including paragraphs:

This is new paragraph. \\ \hline
Dashes: ---, --, {-}, {-}{-}. & URLs: \href{http://pasdoc.sourceforge.net/}{http://pasdoc.sourceforge.net/} \\ \hline
And, last but not least, nested table: 

\begin{tabular}{|l|l|}
\hline
\textbf{1} & \textbf{2} \\ \hline
\textbf{3} & \textbf{4} \\ \hline
\end{tabular}

 & B \\ \hline
\end{tabular}



Some 2{-}row table tests:



\begin{tabular}{|l|l|}
\hline
\textbf{A} & \textbf{B} \\ \hline
\textbf{C} & \textbf{D} \\ \hline
\end{tabular}





\begin{tabular}{|l|l|}
\hline
A & B \\ \hline
C & D \\ \hline
\end{tabular}





\begin{tabular}{|l|l|}
\hline
A & B \\ \hline
\textbf{C} & \textbf{D} \\ \hline
\end{tabular}


\end{document}
