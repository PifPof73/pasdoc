\documentclass{report}
\usepackage{hyperref}
% WARNING: THIS SHOULD BE MODIFIED DEPENDING ON THE LETTER/A4 SIZE
\oddsidemargin 0cm
\evensidemargin 0cm
\marginparsep 0cm
\marginparwidth 0cm
\parindent 0cm
\setlength{\textwidth}{\paperwidth}
\addtolength{\textwidth}{-2in}


% Conditional define to determine if pdf output is used
\newif\ifpdf
\ifx\pdfoutput\undefined
\pdffalse
\else
\pdfoutput=1
\pdftrue
\fi

\ifpdf
  \usepackage[pdftex]{graphicx}
\else
  \usepackage[dvips]{graphicx}
\fi

% Write Document information for pdflatex/pdftex
\ifpdf
\pdfinfo{
 /Author     (Pasdoc)
 /Title      ()
}
\fi


% definitons for warning and note tag
\usepackage[most]{tcolorbox}
\newtcolorbox{tcbwarning}{
 breakable,
 enhanced jigsaw,
 top=0pt,
 bottom=0pt,
 titlerule=0pt,
 bottomtitle=0pt,
 rightrule=0pt,
 toprule=0pt,
 bottomrule=0pt,
 colback=white,
 arc=0pt,
 outer arc=0pt,
 title style={white},
 fonttitle=\color{black}\bfseries,
 left=8pt,
 colframe=red,
 title={Warning:},
}
\newtcolorbox{tcbnote}{
 breakable,
 enhanced jigsaw,
 top=0pt,
 bottom=0pt,
 titlerule=0pt,
 bottomtitle=0pt,
 rightrule=0pt,
 toprule=0pt,
 bottomrule=0pt,
 colback=white,
 arc=0pt,
 outer arc=0pt,
 title style={white},
 fonttitle=\color{black}\bfseries,
 left=8pt,
 colframe=yellow,
 title={Note:},
}

\begin{document}
\label{toc}\tableofcontents
\newpage
% special variable used for calculating some widths.
\newlength{\tmplength}
\chapter{Unit ok{\_}links{\_}to{\_}dot{\_}names}
\label{ok_links_to_dot_names}
\index{ok{\_}links{\_}to{\_}dot{\_}names}
\section{Description}
  

  

  

  \begin{list}{}{
\settowidth{\tmplength}{\textbf{See also}}
\setlength{\itemindent}{0cm}
\setlength{\listparindent}{0cm}
\setlength{\leftmargin}{\evensidemargin}
\addtolength{\leftmargin}{\tmplength}
\settowidth{\labelsep}{X}
\addtolength{\leftmargin}{\labelsep}
\setlength{\labelwidth}{\tmplength}
}
\item[\textbf{See also}]
\begin{description}
\item[test{\_}dot{\_}no{\_}dot.nonexist] 

\item[\begin{ttfamily}test{\_}dot{\_}no{\_}dot.TClass\end{ttfamily}(\ref{test_dot_no_dot.TClass})] 

\item[\begin{ttfamily}test{\_}dot{\_}no{\_}dot.TClass.Field\end{ttfamily}(\ref{test_dot_no_dot.TClass-Field})] 

\item[test{\_}dot.one{\_}dot.nonexist] 

\item[\begin{ttfamily}test{\_}dot.one{\_}dot.TClass\end{ttfamily}(\ref{test_dot.one_dot.TClass})] 

\item[\begin{ttfamily}test{\_}dot.one{\_}dot.TClass.Field\end{ttfamily}(\ref{test_dot.one_dot.TClass-Field})] 

\item[test{\_}dot.two.dots.nonexist] 

\item[\begin{ttfamily}test{\_}dot.two.dots.TClass\end{ttfamily}(\ref{test_dot.two.dots.TClass})] 

\item[\begin{ttfamily}test{\_}dot.two.dots.TClass.Field\end{ttfamily}(\ref{test_dot.two.dots.TClass-Field})] 

\item[test{\_}dot.three.dots.nonexist] 

\item[\begin{ttfamily}test{\_}dot.three.dots.TClass\end{ttfamily}(\ref{test_dot.three.dots.TClass})] 

\item[\begin{ttfamily}test{\_}dot.three.dots.TClass.Field\end{ttfamily}(\ref{test_dot.three.dots.TClass-Field})] 

\end{description}
\end{list}

\chapter{Unit test{\_}dot.one{\_}dot}
\label{test_dot.one_dot}
\index{test{\_}dot.one{\_}dot}
\section{Overview}
\begin{description}
\item[\texttt{\begin{ttfamily}TClass\end{ttfamily} Class}]
\end{description}
\section{Classes, Interfaces, Objects and Records}
\ifpdf
\subsection*{\large{\textbf{TClass Class}}\normalsize\hspace{1ex}\hrulefill}
\else
\subsection*{TClass Class}
\fi
\label{test_dot.one_dot.TClass}
\index{TClass}
\subsubsection*{\large{\textbf{Hierarchy}}\normalsize\hspace{1ex}\hfill}
TClass {$>$} TObject
%%%%Description
\subsubsection*{\large{\textbf{Fields}}\normalsize\hspace{1ex}\hfill}
\begin{list}{}{
\settowidth{\tmplength}{\textbf{Field}}
\setlength{\itemindent}{0cm}
\setlength{\listparindent}{0cm}
\setlength{\leftmargin}{\evensidemargin}
\addtolength{\leftmargin}{\tmplength}
\settowidth{\labelsep}{X}
\addtolength{\leftmargin}{\labelsep}
\setlength{\labelwidth}{\tmplength}
}
\label{test_dot.one_dot.TClass-Field}
\index{Field}
\item[\textbf{Field}\hfill]
\ifpdf
\begin{flushleft}
\fi
\begin{ttfamily}
public Field: byte;\end{ttfamily}

\ifpdf
\end{flushleft}
\fi


\par  \end{list}
\chapter{Unit test{\_}dot.three.dots}
\label{test_dot.three.dots}
\index{test{\_}dot.three.dots}
\section{Overview}
\begin{description}
\item[\texttt{\begin{ttfamily}TClass\end{ttfamily} Class}]
\end{description}
\section{Classes, Interfaces, Objects and Records}
\ifpdf
\subsection*{\large{\textbf{TClass Class}}\normalsize\hspace{1ex}\hrulefill}
\else
\subsection*{TClass Class}
\fi
\label{test_dot.three.dots.TClass}
\index{TClass}
\subsubsection*{\large{\textbf{Hierarchy}}\normalsize\hspace{1ex}\hfill}
TClass {$>$} TObject
%%%%Description
\subsubsection*{\large{\textbf{Fields}}\normalsize\hspace{1ex}\hfill}
\begin{list}{}{
\settowidth{\tmplength}{\textbf{Field}}
\setlength{\itemindent}{0cm}
\setlength{\listparindent}{0cm}
\setlength{\leftmargin}{\evensidemargin}
\addtolength{\leftmargin}{\tmplength}
\settowidth{\labelsep}{X}
\addtolength{\leftmargin}{\labelsep}
\setlength{\labelwidth}{\tmplength}
}
\label{test_dot.three.dots.TClass-Field}
\index{Field}
\item[\textbf{Field}\hfill]
\ifpdf
\begin{flushleft}
\fi
\begin{ttfamily}
public Field: byte;\end{ttfamily}

\ifpdf
\end{flushleft}
\fi


\par  \end{list}
\chapter{Unit test{\_}dot.two.dots}
\label{test_dot.two.dots}
\index{test{\_}dot.two.dots}
\section{Overview}
\begin{description}
\item[\texttt{\begin{ttfamily}TClass\end{ttfamily} Class}]
\end{description}
\section{Classes, Interfaces, Objects and Records}
\ifpdf
\subsection*{\large{\textbf{TClass Class}}\normalsize\hspace{1ex}\hrulefill}
\else
\subsection*{TClass Class}
\fi
\label{test_dot.two.dots.TClass}
\index{TClass}
\subsubsection*{\large{\textbf{Hierarchy}}\normalsize\hspace{1ex}\hfill}
TClass {$>$} TObject
%%%%Description
\subsubsection*{\large{\textbf{Fields}}\normalsize\hspace{1ex}\hfill}
\begin{list}{}{
\settowidth{\tmplength}{\textbf{Field}}
\setlength{\itemindent}{0cm}
\setlength{\listparindent}{0cm}
\setlength{\leftmargin}{\evensidemargin}
\addtolength{\leftmargin}{\tmplength}
\settowidth{\labelsep}{X}
\addtolength{\leftmargin}{\labelsep}
\setlength{\labelwidth}{\tmplength}
}
\label{test_dot.two.dots.TClass-Field}
\index{Field}
\item[\textbf{Field}\hfill]
\ifpdf
\begin{flushleft}
\fi
\begin{ttfamily}
public Field: byte;\end{ttfamily}

\ifpdf
\end{flushleft}
\fi


\par  \end{list}
\chapter{Unit test{\_}dot{\_}no{\_}dot}
\label{test_dot_no_dot}
\index{test{\_}dot{\_}no{\_}dot}
\section{Overview}
\begin{description}
\item[\texttt{\begin{ttfamily}TClass\end{ttfamily} Class}]
\end{description}
\section{Classes, Interfaces, Objects and Records}
\ifpdf
\subsection*{\large{\textbf{TClass Class}}\normalsize\hspace{1ex}\hrulefill}
\else
\subsection*{TClass Class}
\fi
\label{test_dot_no_dot.TClass}
\index{TClass}
\subsubsection*{\large{\textbf{Hierarchy}}\normalsize\hspace{1ex}\hfill}
TClass {$>$} TObject
%%%%Description
\subsubsection*{\large{\textbf{Fields}}\normalsize\hspace{1ex}\hfill}
\begin{list}{}{
\settowidth{\tmplength}{\textbf{Field}}
\setlength{\itemindent}{0cm}
\setlength{\listparindent}{0cm}
\setlength{\leftmargin}{\evensidemargin}
\addtolength{\leftmargin}{\tmplength}
\settowidth{\labelsep}{X}
\addtolength{\leftmargin}{\labelsep}
\setlength{\labelwidth}{\tmplength}
}
\label{test_dot_no_dot.TClass-Field}
\index{Field}
\item[\textbf{Field}\hfill]
\ifpdf
\begin{flushleft}
\fi
\begin{ttfamily}
public Field: byte;\end{ttfamily}

\ifpdf
\end{flushleft}
\fi


\par  \end{list}
\end{document}
